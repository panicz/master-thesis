\section*{Streszczenie}

Celem niniejszej pracy jest wypracowanie technik pozwalających
na efektywne wykonywanie programów napisanych w stylu funkcyjnym.
Praca składa się z dwóch części. W pierwszej przedstawione są klasyczne
techniki transformacji programów funkcyjnych do postaci imperatywnej
oraz podstawowe metody dowodzenia twierdzeń o własnościach programów.
W części drugiej proponowana jest metoda przekształcania pewnych klas
programów operujących na listach w równoważne programy operujące na
tablicach. Ponadto analizowane są warunki pozwalające na przekształcenie
funkcyjnej implementacji algorytmu sortowania szybkiego do optymalnej
postaci imperatywnej.

Wszystkie programy źródłowe oraz transformacje wyrażone są przy pomocy
czysto funkcyjnego podzbioru algorytmicznego języka Scheme, opisa\-nego
w rozdziale 2. Docelowym modelem obliczeń jest wariant maszyny RAM, której
model i zestaw instrukcji zostały dogłębnie opisane w rozdziale 3, wraz
z implementacją, wykorzystującą imperatywne składniki języka Sche\-me.

W rozdziale 4 zaprezentowane są klasyczne techniki
przekształcania programów wyrażonych w omówionym podzbiorze języka
Scheme w ciągi instrukcji dla maszyny RAM, w szczególności konwersja
programów do postaci przekazującej kontynuacje (ang. \textit{Continuation-Passing
  Style}) oraz optymalizacja rekurencji ogonowej (ang. \textit{Tail-Call
  Optimization}).

Rozdział 5 opisuje uproszczony wariant systemu Boyera-Moore'a wraz
z pełną listą aksjomatów służących do dowodzenia twierdzeń o programach
wyrażonych w zaprezentowanym podzbiorze języka Scheme. W przeciwieństwie
jednak do oryginalnego systemu Boyera i Moore'a, wypracowany system nie
jest w stanie samodzielnie dowodzić twierdzeń, i może jedynie służyć
do sprawdzania poprawności dowodów wprowadzonych przez użytkownika.

W rozdziale 6 wypracowana zostaje autorska metoda konwersji programów
funkcyjnych do postaci otrzymujących i przekazujących tablice. Językiem
źródłowym jest czysto funkcyjny podzbiór języka Scheme opisany w rozdziale
2, zaś językiem docelowym -- pełny język Scheme zawierajacy składniki
imperatywne. Wyprawcowana metoda konwersji ma jedynie szkicowy charakter
i z pewnością wymaga dopracowania.

Rozdział 7 podejmuje zagadnienie automatycznej konwersji funkcyjnego
wariantu algorytmu sortowania szybkiego do postaci imperatywnej,
jednak nie prezentuje działającego algorytmu konwersji.

\section*{Słowa kluczowe}
struktury danych, transformacje programów, kompilacja, dowodzenie twierdzeń,
programowanie funkcyjne
