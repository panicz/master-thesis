\chapter{Introduction}

\begin{chapquote}{Lewis Carroll, \textit{Through the Looking Glass}}
  “When I use a word,” Humpty Dumpty said, in rather a scornful tone,
  “it means just what I choose it to mean—neither more nor less.”
\end{chapquote}

\section{Motivation}

Functional programming is a valuable technique for building large
and reusable software systems. Contrasted with more widespread
approaches, its main advantage is that it simplifies the reasoning
about program's behavior. Furthermore, it allows programmers to
provide less detail about the program, making it susceptible to run
in various different setups.

\subsection{Historical perspective -- algorithms}

Historically, programmers had to specify computations in terms
of actions that were to be taken in order to achieve a desired goal.
Initially, specifications were written in languages specific to
particular machines being programmed.

This style of thinking -- i.e. specifying a series of steps to
achieve a certain goal -- gave birth to the branch of Computer
Science known as Algorithm Design.

Being very diverse, machine languages turned out to be inappropriate for
communicating algorithms between programmers of different machines.
For this reason, the FORTRAN and ALGOL families of languages were invented
as means of conveying the ideas of specific algorithms in a precise
and uniform manner\cite{Sebesta2008}. In addition, tools existed that allowed
to run programs written in those languages on real machines, although with
some performance penalty -- therefore, for performance-critical applications,
programmers were still writing machine code\cite{Victor2013}.

What those languages had in common is that their main purpose was
to instruct computer how to perform computations. In addition,
the means of expression of those languages were designed, so that
they could easily be translated directly to machine code of most CPUs.
With time, however, the way of thinking about computers had changed
significantly. No longer were they just mere tools for performing
scientific computations, but they were becoming larger and larger
systems that were expected to run multiple programs simultaneously
in real time.

Furthermore, the complexity of the applications was only increasing,
and the old ways of thinking were often insufficient for managing that
complexity.

As the performance of computers was improving, our requirements
with regard to programming systems begun to shift. For many applications
it was no longer necessary to get the maximum performance of every
CPU cycle, as there were other factors, such as network latency, that
bounded the overall performance of the application.

Moreover, there turned out to be a physical limit on the clock speed
of a single CPU core, which caused hardware manufacturers to focus
on delivering processors with more and more cores of the same speed.
Consequently, the ability to disperse execution of a program on many
cores, or even many machines, can often be more profitable than being
able to perform a few more instructions per second by a sequential
program.

The abundance of processing power had yet another consequence.
As programmers no longer had to worry (too much) about computational
resources, they could experiment more with the means of expression
of their programming languages. After all, it turned out that the two
most costly aspects of computer application development were
(1) programmers' time and (2) programmers' mistakes. It therefore
seems prudent to focus on shortening the time of application development
and on minimizing the possibilities of making mistakes.

\subsection{Programming as expressing ideas}

Eventually, the goal of programming is no longer bending computer
behavior to programmer's will, but describing that will accurately,
precisely and unambiguously. 

This is an interesting process: programmers experiment with programming
languages, which in turn provide them with means to think about various
phenomena, or prompt them to come up with certain ideas. There is
a harmful byproduct of this process though, namely -- the multitude
of programming languages available. Computer programmers really are
in the position of Lewis Carroll's Humpty Dumpty: when they use
a word, it is them who chooses its meaning, and they can make it
mean anything they like. The downside of this freedom is that it
impedes communication, and accordingly -- collaboration.

As a result, a single most valuable trait of a programming language
-- as a means of communication, as opposed to development of particular
applications -- is its simplicity. A good programming language should
be simple to learn, and the programs written in it should be simple
to understand (as simple, one could say, as possible, but no simpler).

Of course, there is nothing in a programming language alone that
can prevent a programmer from using it in an obscure way, and therefore
in addition to the use of appropriate means of expression, programmers
need to obey a certain discipline, or a set of conventions. Needless
to say, the conventions themselves should be clear and simple to follow,
in order to avoid their misinterpretations.

One could ask whether it would be a good idea to use some natural language,
such as English, to express computer programs. The big advantage of this
approach is that most people already know at least one natural language,
their so-called mother tongue.

However, there are some serious drawbacks of that idea. Natural languages
tend to be vague, verbose, imprecise and structurally ambiguous\footnote{
  The idea of using natural language to program computers has been criticized
  at length by Edsger Dijkstra, who concluded that ``machines to be programmed
  in our native tongues —be it Dutch, English, American, French, German,
  or Swahili— are as damned difficult to make as they would be to
  use''\cite{Dijkstra1997}}.
People realized this fact long before the computer has been conceived, and this
realization was reflected in the development of the language of mathematics,
which has complementary advantages to those of natural languages: it is
usually brief, non-ambiguous and precise\footnote{However, certain problems
  with the way the language of mathematics is customarily used to lay out
  physics have been pointed out Gerald Sussman, who proposed to use a programming
  language for that purpose instead\cite{SICM}.}.

Mark Turner and Gilles Fauconnier, the authors of an important cognitive science
book ``The Way We Think'', noted that

\begin{quote}
  The development of formal systems to leverage human invention
  and insight has been a painful, centuries-long process. [...]
  In the twelfth century, the Hindu mathematician Bhaskara said,
  ``The root of the root of the quotient of the greater irrational
  divided by the lesser one being increased by one; the sum being
  squared and multiplied by the smaller irrational quantity is the
  sum of the two surd roots.'' This we would now express in the form
  of an equation, using the much more systematically manageable set
  of formal symbols shown below. This equation by itself looks no
  less opaque than Bhaskara's description, but the notation immediately
  connects it to a large system of such equations in ways that
  make it easy to manipulate.\cite{FauconnierTurner2002}
  \begin{equation*}
    \sqrt{(\sqrt{\frac{n}{k}}+1)^{2}k}=\sqrt{k}+\sqrt{n}
  \end{equation*}
\end{quote}

One could therefore ask whether it might be a better idea to program
computers in the language of mathematics, and indeed some successful
attempts were made in this regard.

However, if we look at how the language of mathematics is usually being
communicated, we find that it is rarely self-contained -- that the concise
mathematical formulas are often interleaved with more lengthy explanations
provided in the form of natural language sentences, whether in text
or near the blackboard. This ``classroom'' or ``handbook'' setup provides
an explanation for the mathematicians' inclination to use short symbols
that are easy to write quickly, although they do not, by themselves,
provide any clues with regard to what they mean or how they should be
pronounced.

There was an attempt to transplant this ``handbook'' approach to
the domain of Computer Science called ``Literate Programming'',
undertaken by the prominent mathematician Donald Knuth\cite{Knuth1992}.
However, the industry practices show that this approach failed to become
widespread.

A likely reason for this state of affairs is that it requires
to maintain the correspondence between the program and its explanation,
and there are no tools that could enforce this correspondence.

\subsection{Referential transparency}

A computer program, when perceived as a speech act, seems to
be liberated from some of the constraints of both natural language
and the language of mathematics, while imposing some of its own
constraints.

Computer programs are usually written on computers, which aid
not only in editing and displaying them, but also in paraphrasing
them (which, in the parlance of programmers, is called
\emph{refactoring}\cite{Fowler2000}) and proving them correct.

If we look at industry practices, we'll notice, that there are also
some things that the natural languages have in common with
the so-called ``industry best practices'': just as there is a great
deal of redundancy in natural languages, that minimizes the risk
of misinterpreting an utterance, programmers are recommended
to augment their programs with redundant features such as tests,
assertions and type
signatures\cite{Feathers} \cite{Fowler2000} \cite{HuntThomas2000}
\cite{Martin2009}.

Another feature that is shared by natural languages and mathematics,
but has only recently been gaining popularity among computer programmers,
is called \emph{referential transparency}, and is closely related to
\emph{the principle of compositionality}.

Referential transparency is a trait of languages' expressions that allows
to substitute a reference to an object with another reference to the same
object (or -- if possible -- with that object itself), yielding
a \emph{co-extensional} expression. For example, in the expression
\begin{equation*}
  2 + 2 = 2 * 2
\end{equation*}
we can replace both left- and right-hand side of the $=$ sign with
some other expressions that has the same value, for example
\begin{equation}
  2 + 2 = 4.
\end{equation}

Of course, this rule applies not only to mathematics, but
to ethnic languages as well. Incidentally, we can replace
the name \emph{William Shakespeare} with the phrase
\emph{the author of Hamlet} in an utterance, usually preserving
its overall meaning.

Referential transparency usually applies to most expressions
of our everyday language. However, there are two situations
where it fails: context-sensitive terms (such as the words \emph{this}
or \emph{you}) and so-called \emph{intensional contexts}\cite{Frege1892}
(for example, we cannot transform the sentence \emph{Lois Lane
knows that Clark Kent is Clark Kent} into \emph{Lois Lane knows
that Clark Kent is Superman}, despite the fact, that both
\emph{Clark Kent} and \emph{Superman} refer to the same entity).

The programming style that employs referential transparency
is called \emph{functional programming}, and it allows to
comprehend the code in terms of the \emph{substitution model
  of computation}\cite{SICP}.

For example, the following Python code that computes the \emph{factorial}
function in the so-called \emph{imperative style} contains identifiers
that are not referentially transparent:

\begin{Snippet}
  def factorial(n):
      accumulator = 1 
      while n > 0:
          accumulator = accumulator * n
          n = n - 1
      return accumulator
\end{Snippet}

In every iteration, the meanings of the names \texttt{accumulator}
and \texttt{n} are different: although they are only used in a single
lexical context, there are many different invisible ``execution contexts'',
in which those names are used. This phenomenon seems to have no counterpart
neither in natural languages nor in the language of mathematics, where
the meanings of words are either fixed, or depend only on some lexical
context\footnote{The author remembers the following riddle from his
  childhood. A mean kid asks another kid (a victim): ``I am me, you are you,
  which of us is the stupid one?''. If the victim responds ``you'', then
  the mean kid claims that -- according to what they settled earlier -- the
  identifier ``you'' refers to the victim, so the victim admitted to
  be the stupid one. On the other hand, if the victim responds ``me'',
  the mean kid is also laughing, because he made the victim admit to
  be stupid.
  Note that in this case, the identifiers ``me'' and ``you'' were
  not modified. They preserved their original meanings, but they were
  shadowed by some new meanings, that referred to the old meanings.
}.

However, the same function can be defined in a referentially transparent
manner with the use of recursion:

\begin{Snippet}
  def factorial(n):
      if n == 0:
          return 1
      else:
          return n * factorial(n - 1)
\end{Snippet}

Since no variable assignment appears in the code, we can evaluate, say,
the expression \texttt{factorial(5)} by substituting the references
to function at given points with their values at that point:

\begin{Snippet}
  >>> factorial(5)
  === 5 * factorial(4)
  === 5 * 4 * factorial(3)
  ...
  === 5 * 4 * 3 * 2 * 1 * 1
  === 120
\end{Snippet}

\subsection{Historical perspective -- data structures}

The advantage of functional programs is that they tend to be more
modular and reusable than their imperative counterparts. From the
point of view of software engineering, they allow only a single way
of passing information, namely -- either by providing it as arguments to
functions, or collecting it from functions' results. The lack of
assignment doesn't allow to pass information through global variables,
for instance, which results in cleaner systems that are easier
to maintain. Also, if programs are written with pure functions,
the order of their evaluation doesn't matter (due to the
Church-Rosser theorem\cite{Harrison1997} \cite{FelleisenFlatt2006}),
which makes it easier to perform evaluations in parallel on different
machines or CPU cores\cite{Backus1977} \cite{Dybvig1987}.

However, this simplicity comes with a price. The users of
imperative languages can tailor their programs to make the best
use of the hardware that's going to be used to run it. Moreover,
there is a considerable collection of solutions readily available
for imperative programming languages, created under the assumption
that \emph{programs = algorithms + data structures}\cite{Wirth1976},
\cite{CLRS}. 

The designers of programming languages quickly realized, that
the \emph{programs = algorithms + data structures} paradigm
imposes an enormous cognitive strain on programmers. If they
wish to use the advantages of a few particular data structures,
they usually need to write code that blends some catalogued data
structures together.

One attempt to solve this problem was to separate data structure
implementations from their interfaces. This idea was embodied
in the \emph{Standard Template Library} of the C++ programming
language, which provided a few different data structures that
could be used with a fixed set of generic functions\cite{Stroustrup1997}
\cite{Stepanov1986}.

It was also at the core of the SQL family of languages, which
actually were designed as a unified interface to a plenitude
of various data structures.

Another idea was to provide a few basic, most commonly used
data structures in the core language. Those typical data structures
usually are: an ordered sequence of elements of arbitrary length
(lists), an ordered sequence of elements of fixed length (tuples),
a mapping between some key values and corresponding target values
(dictionaries, usually implemented using hash tables), and
unordered set of elements of arbitrary size (sets).

The latter approach has been employed by many popular programming
languages, such as Perl, Python, PHP, Ruby or JavaScript. Although
the performance of their versatile built-in data structures
is usually worse than in the case of tailored solutions, it
is often sufficiently good for practical applications, and the greater
conceptual simplicity of the source code makes its development
and maintenance easier.

An extreme version of this approach was proposed in 1958 by John McCarthy,
who designed the LISP programming language\cite{McCarthy1960}. LISP
used one data structure to represent collections, namely -- singly
linked lists. They turned out to be expressive enough to embrace
not only sets and multisets, but also dictionaries (as lists of key-value
pairs).

Since its inception, singly linked lists became a predominant data structure
in functional programming\cite{Bagwell2002}, which -- in conjunction
with the technique of garbage collection (also pioneered by McCarthy)
allowed to develop programs devoid of state mutation.

Unfortunately, singly linked lists have some undesired properties
that disqualify them in a number of applications. For example,
the access time to a random element is linear (as opposed to
constant, as it is the case for arrays), similarly to calculating
the length of a list. Furthermore, modern CPUs are optimized
to process data that is organized in vectors, and the non-locality
of link reference operation may contribute to an increased number
of cache misses, left alone the doubled memory consumption
caused by the need to store additional pointer along with each
element of the list.

For a long time, those deficiencies prevented functional programming
from becoming widespread. Even the programming languages that were
advertised as functional usually incorporated arrays in the repertoire
of their primitive data structures, and specified their order
of evaluation\cite{R5RS} (or provided some sophisticated means
to do so\cite{Wadler1995}) in order to be able to use them
in a predictive way.

In 2002, Phil Bagwell proposed a data structure called VList,
which merged the functionality of linked lists with vectors.
The interface to the data structure remained unchanged compared
to McCarthy's version, so the new data structure could easily
replace linked lists in both compiled and interpreted code\cite{Bagwell2002}.

For exactly the same reason, it didn't provide the improvement
one could hope for -- functions that allocate resources in run time,
without any help from static analysis, must remain ignorant to the
way in which those allocated resources are used.

\section{Formulation}

Let's consider the following program -- the variant of Quick-sort,
written in the Haskell programming language:

\begin{Snippet}
1 qsort [] = []
2 qsort (p:xs) = (qsort below) ++ [p] ++ (qsort above)
3     where
4         below = filter (< p) xs
5         above = filter (>= p) xs
\end{Snippet}

The central idea of the algorithm is expressed in the line 2:
that the resulting sequence consists of the (sorted) elements
smaller than pivot, followed by pivot, and then followed by the
(sorted) elements greater than pivot.

Let's contrast it with the imperative version from \cite{CLRS}:

\begin{codebox}
\Procname{$\proc{Quicksort}(A,p,r)$}
\li \If $p < r$
\li     \Then
           $q \gets \proc{Partition}(A,p,r)$
\li        $\proc{Quicksort}(A,p,q-1)$
\li        $\proc{Quicksort}(A,q+1,r)$
        \End
\end{codebox}

where $\proc{Partition}$ is defined as

\begin{codebox}
\Procname{$\proc{Partition}(A,p,r)$}
\li $x \gets A[r]$
\li $i \gets p-1$
\li \For $j \gets p$ \To $r-1$
\li     \Do \If $A[j] \le x$
\li             \Then $i \gets i+1$
\li                  exchange $A[i] \leftrightarrow A[j]$
\End \End
\li exchange $A[i+1] \leftrightarrow A[r]$
\li \Return i + 1   
\end{codebox}

These definitions come with a remark, that ``to sort an entire array $A$,
the initial call is $\proc{Quicksort}(A,1,length[A])$''.

There are a few things to note here. First, that the imperative definition
is much more difficult to follow, because array indexing adds another
layer of indirection. Also, the imperative version is less composable,
because it modifies its argument $A$, so if the old array ought to be
used in some other part of the program, the programmer has to remember
to make a copy.

On the other hand, the \texttt{qsort} function defined in Haskell
can only superficially be called \emph{quick}: both the \texttt{filter}
function and the list concatenation operator \texttt{++} allocate new
storage, and the time complexity of the \texttt{++} operator is proportional
to the length of its leftmost argument. Furthermore, each use of
the \texttt{filter} function traverses the input list once, so
it may be traversed twice per each invocation of \texttt{qsort}.
By contrast, the $\proc{Partition}$ function traverses
the input array only once.

There is also a significant difference in the content of the data
structures at the intermediate steps of computation: the \texttt{above}
and \texttt{below} lists will contain elements in the order in which
they appear in the \texttt{xs} list, while the result of $\proc{Partition}$
is generally difficult to determine. 

The reason why this question doesn't matter too much is that the elements
eventually end up sorted, no matter what both the initial and intermediate
arrangements of elements are.

Clearly, there is a big difference between the functional \texttt{qsort}
and the imperative $\proc{Quicksort}$, because the latter is based
on a clever idea of $\proc{Partition}$ing an array (in place).

However, we can perceive the Haskell program as a sort of a section through
its imperative counterpart. In particular, it should be possible
to mechanically transform certain classes of functional programs so
that they would use arrays instead of lists, and reuse previously allocated
storage instead of allocating new one. This is what this work is going to
be about.

\section{Structure of this work}

This work is organized in two parts -- the purpose of the first part
is to introduce the basic notions that are going to be needed
to formulate the problem in a bit more rigorous terms. In particular,
chapter 2 describes the source programming language that we wish to use
for expressing our programs, and chapter 3 proposes a simple register
machine model and its instruction set, which is the target of our
transformations/optimizations. In chapter 4, we present a general
transformation which allows to execute programs in our source
language on the target machine (this transformation is traditionally
called \textit{compilation}). Chapter 5 gives an overview of a
system for reasoning about programs expressed in (a subset of)
the source language.

In the second part, we develop some techniques for optimizing
various classes of programs. In chapter 6, we develop a technique
that allows to transform various basic functions that operate on
lists into procedures operating on arrays. Chapter 7 deals mostly
with Quicksort, and although it fails to deliver working solutions,
we hope that it at least arrives at some valuable conclusions.

In every chapter, there is a lot of code illustrating the ideas
being presented or elaborated. The code is the integral part
of the work, and we encourage the reader to actually read it
(rather than skip it). Except for the chapter 5, all the code was
actually tested (and therefore we suspect that programs in chapter 5
almost certainly contain some bugs).

All the code is written using the Scheme programming language.
We've heard that some people find it difficult to comprehend
the programs written in that language because of the multitude
of parentheses that can be used in some more complex expressions.
In our hope that the hesitant readers overcome their repulsion,
we assure that the code is indented in a way that was supposed
to facilitate the comprehension of the programs, rather than
confuse the readers.

We admit that -- although the syntax of the Lisp family of
programming languages turned out to be perfect for our task
(and we challenge the readers to prove us wrong) --
the typesetting of Lisp programs clearly requires some
elaboration.

On the other hand, we have to say that the habit of building
and testing programs bit by bit gave us a lot of confidence
in the quality of our work. In particular, this approach allowed
us to find out that there are some problems with the functional
implementation of Quicksort proposed in \cite{Hudak1986} (cf.
appendix \ref{HudakQuicksort})

\section{Related work}

Our work partially overlaps on some efforts that were undertaken
in order to reduce the burden of garbage collection to compile
time (\cite{Baker1990}, \cite{Chase1987}, \cite{GopinathHennessy1989},
\cite{Hederman1988}, \cite{Hudak1986}) and of interfacing arrays
in functional languages (\cite{Baker1991}, \cite{Wadler1995}).

The idea of using a theorem-proving system for optimizing programs
appears in \cite{FutamuraKonishiGluck2001}, and is -- to an extent
-- explored in \cite{BurstallDarlington1977}.

An alternative approach to the problem of tackling data structures
in functional program -- not by transforming the programs, but
by designing the data structures in such a way that makes them
more suitable in immutable setup -- can be found in \cite{Okasaki1996}
and \cite{Bagwell2002} (among others).

\section{Acknowledgements}

First and foremost, I would like to thank my family and my girlfriend
for supporting my crazy decision of returning to the University to
fill the gap in my education. In particular, I appreciate the patience
of Dorota, that I have been permanently overusing.

I am grateful to my dearest friend Ścisław Dercz vel Michał Stańczyk,
for countless stimulating discussions and his bright critical remarks
regarding this work (sorry for giving out your secret identity, you
uneducated ragtag). 

I am also grateful to the teachers that I've had a pleasure to work
with during this rather lengthy period of my education, in particular
to Wiesław Pawłowski for sharing his fascination in logic, Janusz
Dybizbański and Maciej Dziemiańczuk for all the fun I've had during
their classes, Tomasz Dzido for giving me an opportunity to spread
the ideas of functional programming in the realm of combinatorics,
Paweł Żyliński for appreciating the succinctness of the solutions
that I have been bringing to him, Grzegorz Madejski for encouraging
me to work on \cite{Godek2016}, that in a way was a prelude to this
work, and of course my supervisor Christoph Schwarzweller who provided
me with the perfect conditions for working on this thesis. I also
deeply appreciate the wise advice of Andrzej Szepietowski who
pointed me to him.

I owe a lot to Paulina Śliwka and Alicja Zurita-Kwapińska from
the Dean's office, who showed me so much kindness and support, and
Andrzej Borzyszkowski, whom I have been troubling constantly.

The design of the virtual machine from chapter 3 was directly
inspired by the course in the Theory of Computation that I took
during the year that I've spent studying Philosophy at the University
of Warsaw. I would like to thank Marcin Mostowski for being a demanding
teacher, and for assigning me the task of preparing a note about the
RAM machines \cite{Godek2012} (although the note is in Polish,
I was nicely surprised to have discovered that the same concept
was described in a popular book about philosophy \cite{Dennett2013}).

I also wanted to thank Sophia Gold for sacrificing
her time to read this work, and for pointing me to a new whole realm
of papers on the related subjects (I have to admit that I am extremely
impressed with you as a person), and Carl Eastlund for helping me out
with inductive proofs, and for co-authoring ``The Little Prover''.

To be honest, I could go on and on with the list of people to whom
I am grateful, and whose existence contributed to this work. I don't
think that I would manage to bring this work into existence without
the three years that I've spent at the University of Gdańsk studying
Philosophy, being influenced by a lot of great teachers, including
Stanisław Judycki, Martyna Koszkało, Rafał Urbaniak, Andrzej Leszczyński,
Aleksandra Pawliszyn, Iwona Krupecka, Wojciech Bęben and many others,
as well as my inspiring and enthusiastic friends and colleagues.
It was the best time in my life.

I may have failed to appreciate the value of the time that I've spent
studying Automatic Control and Robotics at the Technical University
of Gdańsk, where -- despite the ``industrial'' hostility of that place
-- I have also met a couple of soul mates, notably Piotr Suchomski,
who taught me to love Mathematics.

I should also mention Radek Potyraj and Andrzej Macuk, who created
the right conditions at the Fellows company, which allowed me to try
to continue the studies. I am grateful to Michał Janke for dragging
me to their company, and I regret that in the end the things didn't
turn out as good as they could have. I appreciate the attitudes of
my colleagues at work, who have been showing interest in this work
and my opinion and expertise, making me feel as someone important.

My apologies to everyone who might have felt that his or her name
is missing here. For purely ecological reasons, I'm unable to do
the justice to all of you here. (This also includes a lot of people
that I never had any opportunity to meet in person, and whose work
I had to admire from the distance, be it spatial or temporal.)

%% However, these deviations do not disturb \emph{The Principle of
%%   Compositionality} from holding almost universally in the language.
%% The Principle of Compositionality claims, that
%% \begin{quote}
%%   The meaning of a complex expression is determined by
%%   its structure and the meanings of its constituents\cite{Szabó2013}.
%% \end{quote}
%% The implications of The Principle of Compositionality are twofold.
%% First, it clarifies that there are only two things to know in order
%% to be able to use a language, namely the meanings of its simplest terms
%% and the allowed ways of composing those terms.
%% Second, it teaches

% programowanie funkcyjne jako uniwersalny zestaw konwencji
% modelowany na podstawie języka naturalnego

% przezroczystość odniesieniowa i zasada kompozycjonalności

% redundancja

% ``cognitive simplicity''

% quicksort w Haskellu oraz wariant z Cormena

% problem z quicksortem w Haskellu -- pamięć tablicowa

% ogólniejszy problem z Haskellem: trudność wnioskowania o złożoności
% czasowej i pamięciowej

% pojęcia podstawowe: podstawieniowy model obliczeń, aplikatywny
% i normalny porządek ewaluacji

