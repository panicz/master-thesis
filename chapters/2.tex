\chapter{The source language}

\begin{chapquote}{Aristotle, \textit{On Interpretation}}
  First we must define the terms ``noun'' and ``verb'', then
  the terms ``denial'' and ``affirmation'', then ``proposition''
  and ``sentence''.
\end{chapquote}

\section{Overview}

In the previous chapter, we presented two variants of seemingly
the same algorithm -- one written in the functional language Haskell,
and the other presented in its classical imperative form.

In this work, we are going to analyze the general conditions
that need to be satisfied in order to be able to transform functional
programs that operate on lists into equivalent imperative programs
that operate on arrays, and search for the means that can be used
in order to detect whether those conditions are actually satisfied.

We shall begin with defining a programming language
that will be used to express algorithms on data structures
that are supposed to be optimized, and specifying a computational
model that will be the target of our transformations.

% Się zarysowuje plan pracy pomału.
% Rozdział 1: wprowadzenie (i sformułowanie problemu)
% Rozdział 2: organon
% - język funkcyjny: podzbiór języka scheme bez call/cc i set!
%   oraz z nieokreślonym porządkiem ewaluacji
% - model ewaluacji: maszyna abstrakcyjna CK; prosty kompilator
% - dowodzenie własności programów komputerowych (Boyer-Moore)
% - wnioskowanie o złożoności funkcji oraz zużyciu zasobów
% Rozdział 3: dowiedzenie pożądanych własności
% Rozdział 4: benczmarki
% Rozdział 5: podsumowanie i wnioski

%% \section{A purely functional subset of Scheme}

An ideal programming language for our purpose would possess the
following traits: it would be simple to describe, simple to process
and powerful enough to express any computable function.

For those reasons, we decided to choose a purely functional subset
of the Scheme programming language\cite{R5RS}, devoid of the \texttt{set!}
instruction and the \texttt{call-with-current-continuation} control
operator. Furthermore, although the specification of Scheme defines the strict
(or applicative) order of evaluation, the programs presented in this
work shall never rely on this. Lastly, as the point of this work
is to describe techniques for optimizing programs that operate on lists,
our subset of Scheme need not support vectors, in spite of their presence
in the specification. Likewise, we shall ignore the support of character
strings in the language, as well as its numerical tower (for our purpose,
the support for integer numbers should be sufficient).

There's plenty of excellent education resources available for the Scheme
programming language \cite{SICP} \cite{HTDP} \cite{EOPL}
\cite{FriedmanSpringer1993} \cite{FriedmanFelleisen1996} \cite{Dybvig2009}
\cite{Sitaram2003} \cite{Godek2016}, so only a brief introduction to
the language will be presented here.

As Scheme is not only going to be the source language, but also -- occasionally
-- the intermediate language for compilation, we are also going to describe
some imperative features that are going to appear in the generated code.

\section{Syntax}

Syntactically, Scheme employs a fully parenthesized prefix notation
built around the so called s-expressions that could be described
with the following BNF-style grammar:
\begin{Snippet}
<s-expression> ::= <atom> | (<s-expression> . <s-expression>);

<atom> ::= () | <symbol> | <number> | #true | #false;
\end{Snippet}

A \texttt{<number>} is a sequence of decimal digits. A \texttt{<symbol>}
is a sequence of non-whitespace and non-parenthetic characters that
does not begin with \texttt{\#} and cannot be interpreted as
a \texttt{<number>}.

The tokens are separated either by white spaces or by parentheses.

In addition, the \textit{pair notation}, e.g. 
\texttt{(a\,\,.\,\,(b\,\,.\,\,(c\,\,.\,\,d)))}
is equivalent to \texttt{(a\,\,b\,\,c\,\,.\,\,d)}, and in particular
\texttt{(a\,\,.\,\,(b\,\,.\,\,(c\,\,.\,\,())))} (the \textit{last tail} being an
empty list) is equivalent to \texttt{(a\,\,\,b\,\,\,c)}.

The expressions \texttt{(quote \textit{x})},
\texttt{(quasiquote \textit{x})}, \texttt{(unquote \textit{x})}
and \texttt{(unquote\--splicing \textit{x})} can be abbreviated
as \texttt{'\textit{x}}, \texttt{`\textit{x}}, \texttt{,\textit{x}}
and \texttt{,@\textit{x}}, respectively.

The semicolon character (\texttt{;}) is used to mark comments that span to the
end of line, or -- if the semicolon is directly preceded by the \texttt{\#}
character -- until the end of the following s-expression.

\section{Semantics}

The value of a number is that number itself (thus we say that numbers
are self-evaluating). The value of a symbol is the value that has been
bound to that symbol. A symbol is said to be bound if either:
\begin{itemize}
\item it is predefined by the language,
\item it has been defined using a \texttt{define} special form (or
  its derivative) in the current lexical context,
\item it appears in the argument list of a surrounding \texttt{lambda}
  expression (or its derivative).
\end{itemize}
If a symbol is unbound, the meaning of a program is undefined.

An s-expression of the form \texttt{(operator operands ...)}
is called a \textit{combination}. A combination can either
be a special form, or a function application.

\subsection{Special forms}

Special forms can either be primitive or derivative.

\subsection{The primitive special forms}

The subset of Scheme of our interest contains the following
primitive special forms:
\texttt{lambda}, \texttt{define}, \texttt{if} and \texttt{quote}.

\subsubsection{The \texttt{lambda} form}

The \texttt{lambda} form is used to create function, and consists
of two parts: a \textit{list of arguments} and a \textit{body}.
For example, a function of two arguments whose value
is its first argument can be created in the following way:

\begin{equation*}
  \mathtt{(lambda}
  \underbrace{\mathtt{(x\,\,y)}}_{arguments}
  \underbrace{\mathtt{x}}_{body}
  \mathtt{)}
\end{equation*}

Note that the symbols that appear in the argument list must be unique,
i.e. it must not be the case that the same symbol appears on the list
more than once. Also, the body must be a valid Scheme program.

The list of arguments need not be proper. In such cases, the
dotted tail argument represents a list of optional (variadic)
arguments.

Although the \texttt{lambda} form is sufficient to express
any computation \cite{Harrison1997}, including operations
on natural numbers as well as lists\cite{SICP}, Scheme provides
some additional primitive forms for convenience.

\subsubsection{The \texttt{define} form}

The \texttt{define} form is used to extend the current lexical context
with a new binding. It takes two operands:
a symbol that shall be bound with a value (\textit{definiendum}),
and an expression whose value shall be bound to the symbol
(\textit{definiens}). For example, the definition

\begin{equation*}
  \mathtt{(define}
  \underbrace{\mathtt{x}}_{definiendum}
  \underbrace{\mathtt{5}}_{definiens}
  \mathtt{)}
\end{equation*}

causes \texttt{x} to be bound to the value \texttt{5}.
The \texttt{define} special form is the only primitive special form
that is not considered an expression.

Also, since -- unlike \texttt{lambda} -- the \texttt{define} form
does not create a new scope, but rather extends the current one,
it allows to define functions that are recursive.

Although the basic form of \texttt{define} is \texttt{(define name value)},
we are going to treat usages like \texttt{(define (f x) ...)} as short-hand
for \texttt{(define f (lambda (x) ...))} and so on\footnote{In particular,
  the \texttt{(define ((f x) y) ...)} can be treated as a short-hand for
  \texttt{(define f (lambda (x) (lambda (y) ...)))}. This generalized
  feature, called \textit{curried definitions}, is not provided by most
  implementations of Scheme.}.

\subsubsection{The \texttt{if} form}

The \texttt{if} expressions take the form
\begin{Snippet}
  (if <condition> <consequent> <alternative>)
\end{Snippet}
If the \texttt{<condition>} evaluates to \texttt{\#false},
then the value of the whole expression is value of
\texttt{<alternative>}; otherwise, it is the value of
\texttt{<consequent>} (therefore, every value other than
\texttt{\#false} is considered to be true in the context
of an \texttt{if} expression. Such values will be referred
to as \textit{truth-ish} throughout this text).

\subsubsection{The \texttt{quote} form}

The \texttt{quote} form is used to input literal data.
For example, the value of the expression \texttt{(quote x)}
is the symbol \texttt{x}. The \texttt{quote} operator is
redundant for self-evaluating expressions (e.g. numbers),
so there is no practical difference between expressions
\texttt{(quote 1)} and \texttt{1}.

The operator is not idempotent, though: the value of
the expression \texttt{(quote\,\,(quote\,\,1))} is a list
of two elements, whose first element is the symbol
\texttt{quote}, and whose second element is
the number \texttt{1}.

As it was noted in the section describing the syntax,
the expression \texttt{(quote\,\,x)} can be abbreviated
as \texttt{'x}, so -- consequently -- the expression
\texttt{(quote\,\,(quote\,\,1))} can equivalently be written
as \texttt{'\phantom{}'1}, and also \texttt{'(quote\,\,1)}
and \texttt{(quote\,\,'1)}.

\subsubsection{The \texttt{begin} and \texttt{set!} forms}

Although the source programs that we are going to write
are purely applicative, the resulting program will occasionally
contain some procedural constructs. The \texttt{begin} form
is used for sequencing operations. Its value is the value of
the last expression in that form (in particular, \texttt{(begin x)}
and \texttt{x} are equivalent).

The \texttt{set!}\,\,form is used for performing assignment.
For example, the value of the expression\footnote{
  It may be questionable whether that program can actually
  be called an expression, as it introduces a definition
  into its current scope. The snippet presents a very
  bad style of programming, but it also presents the meaning
  of the special forms discussed in this section.
}

\begin{Snippet}
  (begin
    (define x 5)
    (set! x (+ x 1))
    (* x x))
\end{Snippet}

is the number 36. There is no need to use \texttt{begin} in
the body of the \texttt{lambda} form: \texttt{(lambda args (begin actions ...))}
and \texttt{(lambda args actions ...)} are equivalent. Most typically,
the \texttt{begin} form is used in one or both branches of the \texttt{if}
form.

\subsubsection{The derivative forms}

Scheme offers a mechanism that allows to define
derivative special forms, often refereed to as syntax extensions
or macros. It also predefines a set of helpful derivative
special forms: \texttt{let}, \texttt{let*}, \texttt{and},
\texttt{or}, \texttt{cond} and \texttt{quasiquote}.

\subsubsection{The \texttt{let} form}

The \texttt{let} form is used to create local bindings,
and is defined so that
\begin{Snippet}
(let ((<variable1> <value1>)
      (<variable2> <value2>)
      ...)
  body)
\end{Snippet}
expands to
\begin{Snippet}
((lambda (<variable1> <variable2> ...) body) <value1> <value2> ...)
\end{Snippet}

\subsubsection{The \texttt{let*} form}

In the case of the \texttt{let} form, the bindings of variables
to all values occur simultaneously, so for example in the expression
\begin{Snippet}
1 (let ((x 2)
2       (y (+ x 1)))
3   (+ x y))
\end{Snippet}
the symbol \texttt{x} in the line \texttt{2} does not refer to the value
from the binding in the line \texttt{1}, but to the value from the outer
scope of the expression. This behavior is often undesired in practice.

For that reason, Scheme provides a sequential variant of \texttt{let}
called \texttt{let*}, which is defined so that
\begin{Snippet}
(let* ((<variable1> <value1>)
       (<variable2> <value2>)
       ...)
  <body>)
\end{Snippet}
expands to
\begin{Snippet}
(let ((<variable1> <value1>))
  (let* ((<variable2> <value2>)
         ...)
    <body>))
\end{Snippet}

\subsubsection{The \texttt{and} form}

The \texttt{and} form expresses logical conjunction that uses
short-circuited evaluation \cite{McCarthy1960} \cite{Sebesta2008},
defined so that
\begin{Snippet}
(and <condition1> <condition2> ...)
\end{Snippet}
expands to
\begin{Snippet}
(if <condition1>
    (and <condition2> ...)
    #false)
\end{Snippet}
and
\begin{Snippet}
(and <final-condition>)
\end{Snippet}
expands to
\begin{Snippet}
<final-condition>
\end{Snippet}
thereby causing the \texttt{and} clause to expand either to \texttt{\#false}
or to the value of its \texttt{<final-condition>}.

\subsubsection{The \texttt{or} form}

Like the \texttt{and} form, the \texttt{or} special form performs
evaluation in a short-circuited manner, and is also defined to evaluate
to the value of its succeeding clause. This last requirement causes
a slight complication under the strict model of evaluation, as it
requires to capture the value of expression in order to make sure that it
is evaluated only once:

\begin{Snippet}
(or <condition1> <condition2> ...)
\end{Snippet}

expands to 

\begin{Snippet}
(let ((##result <condition1>))
  (if ##result ##result (or <condition2> ...)))
\end{Snippet}

where \texttt{\#\#result} is a unique identifier that does not appear
anywhere in the rest of the code. As in the case of \texttt{and},
the expression \texttt{(or <final-condition>)} simply expands to
\texttt{<final-condition>}.

\subsubsection{The \texttt{cond} form}

The \texttt{cond} form is used to avoid nested \texttt{if}s\footnote{Actually,
  the \texttt{cond} form present in the Scheme language is a bit more complex
  than presented here.}.
The expression
\begin{Snippet}
(cond (<condition1>
       <value1>)
      (<condition2>
       <value2>)
       ...)
\end{Snippet}
gets expanded to
\begin{Snippet}
(if <condition1>
    <value1>
    (cond (<condition2>
           <value2>)
          ...))
\end{Snippet}

\subsubsection{The \texttt{when} and \texttt{unless} forms}

In the generated imperative code, we may sometimes want to use
the forms \texttt{when} and \texttt{unless}. The former is
defined so that
\begin{Snippet}
  (when condition actions ...)
\end{Snippet}
expands to
\begin{Snippet}
  (if condition (begin actions ...))
\end{Snippet}
and the latter -- so that
\begin{Snippet}
  (unless condition actions ...)
\end{Snippet}
expands to
\begin{Snippet}
  (if (not condition) (begin actions ...))
\end{Snippet}

\subsubsection{The \texttt{quasiquote} form}

The \texttt{quasiquote} form (in conjunction with two helper keywords,
namely -- \texttt{unquote} and \texttt{unquote-splicing}) is used
for creating data conveniently. The convenience stems from the fact,
that the syntaxes \texttt{(quasiquote x)}, \texttt{(unquote x)} and
\texttt{(unquote-splicing x)} can be abbreviated as \texttt{`x},
\texttt{,x} and \texttt{,@x}, respectively. This allows to create
data that contains some variable elements in it, for example the
value of the expression
\begin{Snippet}
(let ((x 3)
      (y '(5 6 7)))
  `(1 2 ,x 4 ,@y 8))
\end{Snippet}
is the list \texttt{(1 2 3 4 5 6 7 8)}.

\subsubsection{Non-standard derivative forms}

\subsubsection{Receiving multiple values}

Scheme allows functions to return more than one value using the
\texttt{values} form. For example, the form \texttt{(values 1 2 3)}
returns three values: \texttt{1}, \texttt{2} and \texttt{3}.

Normally (e.g. in the context of a function call) only the first
value is taken into account, and the remaining ones are ignored.
Scheme provides a special form \texttt{call-with-values},
which allows to capture the remaining return values. For example,
\texttt{(call-with-values (lambda () (values 1 2 3)) list)} passes
the values \texttt{1}, \texttt{2} and \texttt{3} as subsequent argument
to the \texttt{list} function (effectively creating the list \texttt{(1 2 3)}).

We are going to assume here, that the \texttt{let*} form can be abused
to receive multiple values\cite{SRFI-71}, so that, for example

\begin{Snippet}
(let* ((a b c (values 1 2 3)))
  (list a b c))
\end{Snippet}

is equivalent to

\begin{Snippet}
(call-with-values
  (lambda () (values 1 2 3))
  (lambda (a b c) (list a b c)))
\end{Snippet}

Note that the multiple return value feature will never be used
in this work when Scheme is to be treated as the subject language
(i.e. as data to be processed by compilers, interpreters, theorem
provers and other transformers).

\subsubsection{The \texttt{is} form}

We shall also be using a non-standard extension to the Scheme language,
proposed by the author of this work in \cite{SRFI-156}. In short,
it introduces the \texttt{is} special form, such that

\begin{Snippet}
  (is a related-to? b)
\end{Snippet}

expands to

\begin{Snippet}
  (related-to? a b)
\end{Snippet}

In addition, it treats the \texttt{\_} (underscore) symbol in the argument
position specially. For example,

\begin{Snippet}
  (is _ related-to? x)
\end{Snippet}

expands to

\begin{Snippet}
  (lambda (_) (related-to? _ x))
\end{Snippet}

\subsection{Function applications}

If a combination is not a special form, it is treated as a function
application. A function that is applied to can either be a primitive
function, or a function created with a \texttt{lambda} form.

\subsubsection{Functions created with a \texttt{lambda} form}

As noted before, the \texttt{lambda} form contains a list of arguments
and a nested Scheme expression, referred to as the body of the
\texttt{lambda} form.

The value of the application of a function created by a \texttt{lambda}
form is simply the value obtained by evaluating its body in the lexical
scope extended with bindings of its arguments to the values of the
operands of the combination.

For example, the value of the expression
\begin{Snippet}
((lambda (x y) x) 5 10)
\end{Snippet}
is the value of the expression \texttt{x} in the environment where
the symbol \texttt{x} is bound to \texttt{5} and the symbol \texttt{y}
is bound to \texttt{10}.

\subsection{A note on lexical scoping}

Scheme is said to be a \emph{lexically scoped} language, as opposed
to \emph{dynamically scoped}. \emph{Lexical scope} means
that a symbol which occurs free in the body of a \texttt{lambda}
expression (i.e. it does not appear on the argument list of the
\texttt{lambda} expression) refers to the meaning from the context
of the definition of that \texttt{lambda} expression, rather than
the context of its usage (as it is the case for dynamic scope).

For example, if we
\begin{Snippet}
(define square (lambda (x) (* x x)))
\end{Snippet}
it is apparent that the symbol \texttt{*} occurs free in the
definiens. It is typically bound to a procedure that multiplies
its arguments (see \emph{Primitive functions} section below).

The question is, what would be the value of the expression
\begin{Snippet}
(let ((* +))
  (square 5))
\end{Snippet}

where the symbol \texttt{+} is bound to a procedure that adds
its arguments.

In a dynamically scoped language, the value of that expression
is \texttt{10}, whereas in a lexically scoped language it is \texttt{25}.

\subsection{A note on recursive definitions}

Note that there is a significant difference between the bindings created
using the \texttt{define} form and the ones obtained introduced by
\texttt{let} or \texttt{lambda}. In particular, \texttt{lambda} creates
a new scope, whereas \texttt{define} introduces new bindings to the
current scope.

This difference becomes particularly significant if one attempts
to define a recursive function. Because of that, the following
expression of the \textit{factorial} won't work as one could hope:

\begin{Snippet}
(let ((! (lambda (n)
           (if (= n 0)
             1
             (* n (! (- n 1)))))))
  (! 5))
\end{Snippet}

The \texttt{!} symbol in the body of the \texttt{lambda} expression
refers to the binding in the environment in which the \texttt{lambda}
expression is evaluated (the so-called \emph{top-level} environment),
and not the environment introduced by the \texttt{let} form (in which
the \texttt{!} symbol gets bound to the value of the \texttt{lambda}
expression).

This is not the case with the \texttt{define} forms. The \texttt{define}
form introduces a new binding to the current environment, so the
definition

\begin{Snippet}
(define ! (lambda (n)
            (if (= n 0)
              1
              (* n (! (- n 1))))))
\end{Snippet}

actually allows to compute a factorial. It can be noted
however\cite{Harrison1997}, that the \texttt{lambda} form is sufficient
not only for naming intermediate values of computation, but also for
expressing recursive functions, through the application of the so-called
\textit{fixed-point combinators}, i.e. a family of operators $\nabla$, such
that for any $y$, $\nabla\,y\,=\,y\,\nabla\,y$.

An example of a fixed-point combinator is called \texttt{Y}-combinator, and is
defined as
\begin{Snippet}
(define Y (lambda (f)
            (let ((R (lambda (x) (f (x x)))))
              (R R))))
\end{Snippet}

Using this combinator, we can express the recursive factorial function
without the use of the \texttt{define} form (i.e. using only derivatives
of the \texttt{lambda} form) in the following way:

\begin{Snippet}
(let* ((F (lambda (f)
            (lambda (n)
              (if (= n 0)
                1
                (* n (f (- n 1)))))))
       (! (Y F)))
  (! 5))
\end{Snippet}

Such interpretation should work fine in lazy or call-by-name languages
(such as Haskell or Lazy Racket), but will fail in languages with strict
semantics (such as Scheme or Python or JavaScript or Java), that evaluate
their arguments prior to function application, because the \texttt{Y}
combinator would expand \textit{ad infinitum}\footnote{
  Note that it is typical for strict languages to support assignment,
  which allows to express recursion rather easily, e.g.: \\
  \texttt{
    (let ((!\,\,\#false))\\*
    \-\ \ (set!\,\,!\,\,(lambda (n)\\*
    \-\ \ \ \ \ \ \ \ \ \ \ \ (if (= n 0)\\*
    \-\ \ \ \ \ \ \ \ \ \ \ \ \ \ 1\\*
    \-\ \ \ \ \ \ \ \ \ \ \ \ \ \ (* n (!\,\,(- n 1))))))\\*
    \-\ \ (!\,\,5)) \\*
  }
  Since the \texttt{lambda} expression is created in the context
  where the variable \texttt{!} is already defined (as the value
  \texttt{\#false}), there is no problem with it referring to itself.
}.

Therefore we need a fixed point combinator that yields a \textit{thunk}
(i.e. a function of no arguments), and that this thunk is evaluated
before the recursive call, so that the expanded version of our factorial
would look like this (note the additional pair of parentheses around
the nested \texttt{f} and generating \texttt{(Z F)}):

\begin{Snippet}
(let* ((F (lambda (f)
            (lambda (n)
              (if (= n 0)
                1
                (* n ((f) (- n 1)))))))
       (! ((Z F))))
  (! 5))
\end{Snippet}

where the fixed point combinator \texttt{Z} is defined (non-recursively) as

\begin{Snippet}
(define Z (lambda (f)
            (let ((R (lambda (x)
                       (lambda () (f (x x))))))
              (R R))))
\end{Snippet}

While this construction's purpose is to implement recursion in strict
languages, it should work in lazy languages as well.

\subsection{Primitive functions}

The considered subset of the Scheme programming language also
contains a handful of primitive functions that operate on the
primitive data types, i.e. numbers, lists and symbols.

\subsubsection{Primitive functions that operate on numbers}

The set of primitive functions that operate on numbers shouldn't
be surprising: it consists of functions such as addition (\texttt{+}),
multiplication (\texttt{*}), division (\texttt{/}) and subtraction
(\texttt{-}). The functions \texttt{*} and \texttt{+} accept
arbitrary number or arguments (for zero arguments, they evaluate
to the neutral element of multiplication and addition, respectively).

The functions \texttt{/} and \texttt{-} can take one or more arguments.
The combinations \texttt{(/ x)} and \texttt{(- x)} are equivalent to
\texttt{(/ 1 x)} and \texttt{(- 0 x)}, which evaluate to the inverted
and negated value of \texttt{x}, respectively.

The values of the expressions \texttt{(-\,\,a$_1$\,\,a$_2$\,\,...\,\,a$_n$)}
and \texttt{(/\,\,b$_1$\,\,b$_2$\,\,...\,\,b$_k$)} (where $\mathtt{b}_i \ne 0$
for $1 < i \le k$) are the numbers
$((... (\mathtt{a}_1 - \mathtt{a}_2) - ...) - \mathtt{a}_n)$
and $((... (\mathtt{b}_1 / \mathtt{b}_2) / ...) / \mathtt{b}_k)$.

In addition, there are binary functions \texttt{quotient} and
\texttt{remainder} whose names are rather self-explanatory.

The unary predicate \texttt{number?} can be used to check whether
a given object is a number.

There is also the \texttt{random} function that takes an integer
\texttt{n} and evaluates to a random integer between \texttt{0}
and \texttt{n-1}. Note that although programs that contain
a call to the \texttt{random} function are no longer referentially
transparent, they can still be analyzed in terms of the substitution
model of computation.

\subsubsection{Primitive functions that operate on lists}

The basic function for constructing lists is called \texttt{cons}.
It takes two arguments. The value of the expression
\texttt{(cons \emph{a} \emph{b})} is a (typically) newly allocated
pair \texttt{(\emph{a}\,\,.\,\,\emph{b})} (also called a
\textit{cons cell}) from the garbage-collected heap.

In order to retrieve the values from a cons cell, one can
use the accessor functions \texttt{car} and \texttt{cdr}.
In particular, for any values \texttt{\textit{a}} and
\texttt{\textit{b}}, the expression \texttt{(car (cons \textit{a} \textit{b}))}
evaluates to \texttt{\textit{a}} and the expression
\texttt{(cdr (cons \textit{a} \textit{b}))} evaluates
to \texttt{\textit{b}}.

The predicate \texttt{pair?} can be used to check whether a given
object is a cons cell, so for any values \texttt{\textit{a}}
and \texttt{\textit{b}}, the expression
\texttt{(pair?\,\,(cons\,\,\textit{a}\,\,\textit{b}))}
evaluates to truth-ish value. On the other hand,
for any value \texttt{\textit{x}}, if \texttt{\textit{x}} is an
atom, then \texttt{(pair?\,\,\textit{x})} evaluates
to \texttt{\#false}.

The \texttt{apply} function can be used to apply a function
to a list of arguments. In particular, if \texttt{l} is a list
\texttt{(a$_1$\,\,a$_2$\,\,...\,\,a$_n$)}, then \texttt{(apply\,\,f\,\,l)}
is equivalent to \texttt{(f\,\,a$_1$\,\,a$_2$\,\,...\,\,a$_n$)}.

\subsubsection{Primitive comparison predicates}

The notion of identity in Scheme may seem a bit complex at first.
The primitive predicate \texttt{eq?} can be used to check whether
two expressions evaluate to the same object, that is, an object
that occupies the same space in the computer memory.

In particular, it need not be the case that
\texttt{(eq?\,\,(cons\,\,1\,\,2)\,\,(cons\,\,1\,\,2))}, because
the evaluation of each expression \texttt{(cons\,\,1\,\,2)} may allocate
new memory, instead of re-using the already allocated.

On the other hand, it is guaranteed that two instances of
a symbol with the same shape are \texttt{eq?}, for example
\texttt{(eq?\,\,'abc\,\,'abc)} evaluates to a truth-ish value,
and that two symbols with different shapes are not \texttt{eq?},
so for example \texttt{(eq?\,\,'abc\,\,'def)} evaluates to
\texttt{\#false}. Also, it is guaranteed that the empty list,
i.e. \texttt{'()}, is always \texttt{eq?} to itself.

Situation gets more complicated in the case of numbers.
Typically, two instances of the same number can be \texttt{eq?}
if the numbers are small enough to fit into machine word.
However, since Scheme supports arbitrary precision arithmetic,
additional storage may be allocated on the heap to store
results of arithmetic operations. In such cases, two instances
of the same number may not be \texttt{eq?}.

This is why Scheme provides a separate predicate, \texttt{=},
that is used for checking numerical equality. In addition, it
provides predicates \texttt{<}, \texttt{<=}, \texttt{>} and
\texttt{>=} to check whether their arguments form ascending,
non-descending, descending and non-ascending sequences, respectively.

\subsection{Non-primitive functions defined in the language}

In addition to the primitive functions, Scheme comes with a set
of functions that are predefined, although they can be easily
defined in terms of the primitives.

\subsubsection{The notion of identity}

The notion of identity based on memory address is very difficult
to reason about, and its applicability is very limited. There
is a much more natural and deterministic notion of identity that
can be defined in terms of the presented primitive functions:

\begin{Snippet}
(define (equal? a b)
  (or (eq? a b)
      (and (pair? a)
           (pair? b)
           (equal? (car a) (car b))
           (equal? (cdr a) (cdr b)))
      (and (number? a)
           (number? b)
           (= a b))))
\end{Snippet}

The above definition reads as follows. Two objects are \texttt{equal?}
either if they are \texttt{eq?}, or if they are both \texttt{pair?},
their \texttt{car}s are \texttt{equal?} and \texttt{cdr}s are \texttt{equal?},
or they are both \texttt{number?} and they are \texttt{=}
(numerically equal).

The actual \texttt{equal?} function available in Scheme is a bit more
powerful, as it can take arbitrarily many arguments, and evaluates to
\texttt{\#false} if any two differ (in the sense implied by the above
definition) from each other.

%% \subsubsection{Nonprintable symbols}

%% It is sometimes desirable to be able to generate unique symbols that
%% are distinct from every other object, and in particular -- distinct
%% from every object that can be represented in the source code (they
%% can be useful, for instance, as the value of the \texttt{\#\#result}
%% in the expansion of the \texttt{or} special form presented earlier).

%% We could therefore assume that there is a procedure \texttt{unique}
%% available, whose each invocation returns a new symbol that is not
%% representable in the Scheme source code.

%% Note that the presence of a ``function'' whose value varies with
%% each invocation is a departure from the idea of purely functional
%% programming, although our programs can still be analyzed in terms
%% of the substitution model of computation.

\subsubsection{List processing}

The \texttt{list} function evaluates to a list of its (evaluated) arguments,
for example \texttt{(list\,\,1\,\,(*\,\,1\,\,2)\,(+\,\,1\,\,(*\,\,1\,\,2)))}
evaluates to the list \texttt{(1\,\,2\,\,3)}. The \texttt{list} function
could be defined simply as \texttt{(lambda x x)}, because if the list
of arguments of a \texttt{lambda} expression is improper, then they
are captured in a list and bound with the dotted tail of the argument list.

The \texttt{map} function takes a $n$-ary function \texttt{f} (for $n \ge 1$)
and $n$ lists of length $k$ and returns a new list of length $k$
such that its elements are obtained from application of \texttt{f}
to the subsequent $n$-tuples from the list. In other words, given lists
$L^{(1)}, L^{(2)}, ..., L^{(n)}$, where
$L^{(i)}=(a_1^{(i)}\,a_2^{(i)}\,...\,\,a_k^{(i)})$,
\texttt{(map f $L^{(1)}\,L^{(2)}\,...\,\,L^{(n)}$)} produces a list
\texttt{
  ((f\,$a_1^{(1)} a_1^{(2)} ...\,\,a_1^{(n)}$)
  (f\,$a_2^{(1)} a_2^{(2)} ...\,\,a_2^{(n)}$)\,...
  (f\,$a_k^{(1)} a_k^{(2)}\,...\,\,a_k^{(n)}$))}.

Of course, in the simplest (and most frequently used) case, i.e. when ${n=1}$
and $L = (a_1\,a_2\,...\,\,a_k)$, the expression \texttt{(map f $L$)}
evaluates to a list \texttt{((f\,$a_1$)\,(f\,$a_2$)\,...\,(f\,$a_k$))}.

Another frequently used function is called \texttt{filter}. It takes
a predicate \texttt{p?} and a list \texttt{L}, and returns a new list
that contains only those elements from \texttt{L} that satisfy the
predicate \texttt{p?}. The original order of elements is preserved.
The expression \texttt{(filter\,\,p?\,\,L)} can therefore be read
as ``such elements of \texttt{L} that \texttt{p?}'', for example,
the expression \texttt{(filter (is \_ > 5) L)} can be read
as ``such elements of \texttt{L} that are greater than \texttt{5}''.

\subsubsection{Non-standard functions}

There are a few functions that are going to be used in this work
that are not a part of the standard Scheme. They are explained
here briefly, and their definitions are given in the appendix \ref{non-standard-functions}.

\subsubsection{Quantifiers}

It is common in logic and mathematics to express statements using
the quantifiers \textit{for all} ($\forall$) and \textit{exists} ($\exists$).
For example, the sentence ``All men are mortal'' could be translated
to predicate calculus as $\forall_x[man(x) => mortal(x)]$. We usually
assume that there is a domain (universe) of all objects that we can talk
about.

In Scheme, we usually have to be a bit more specific. It is customary
to define two functions, usually called \texttt{every} and \texttt{any},
that take a predicate and a list of objects from the domain, and evaluate
to \texttt{\#false} if every object from the list satisfies the
predicate (in case of \texttt{every}) or there is no object in the list
that satisfies the predicate (in case of \texttt{and}), and otherwise
evaluate to some truth-ish value.

So while the most faithful translation of the above example to Scheme
would have the form
\begin{Snippet}
  (every (lambda (x)
           (if (man? x)
             (mortal? x)
           ;else
             #true))
    THE-UNIVERSE)
\end{Snippet}

it is much more customary to assume that we have a list of all men,
and check whether each element of that list satisfies the predicate
\texttt{mortal?}, i.e. \texttt{(every\,\,mortal?\,\,men)}.

\subsubsection{Set operators}

Lists can interpreted not only as sequences, but also as sets.
On such occasions, the order of elements on the list becomes
immaterial. An \texttt{element} is thought of as being a member of
a set represented by a list if it is a \texttt{member} of that list,
which is denoted as \texttt{(member element list)}
or \texttt{(is element member list)}.

One can easily define the usual set operators such as \texttt{union},
\texttt{intersection} and \texttt{difference}. However, the order
of elements contained in the result of these operations is undefined
and should not be relied on (in particular, it might be the case
that any of these functions called with the same arguments return
the result in a different order upon each invocation).

\subsubsection{Folding}

Suppose that we are given a list, for example $[a_1,a_2,a_3,...]$,
and a binary operator $\circ$. The simplest variant of the $fold$
operation computes the value of the expression
$(a_1 \circ a_2 \circ a_3 \circ ...)$.

Note that the above formulation is ambiguous. For example,
\begin{equation*}
  fold(\circ, [a_1,a_2,a_3,a_4])
\end{equation*}
i.e.
\begin{equation*}
  (a_1 \circ a_2 \circ a_3 \circ a_4)
\end{equation*}
can be interpreted as either
\begin{equation} \label{fold-left}
  (((a_1 \circ a_2) \circ a_3) \circ a_4)
\end{equation}
or
\begin{equation} \label{fold-right}
  (a_1 \circ (a_2 \circ (a_3 \circ a_4)))
\end{equation}
or
\begin{equation} \label{fold-tree}
  ((a_1 \circ a_2)\circ(a_3 \circ a_4)).
\end{equation}

The interpretation (\ref{fold-left}) is called \emph{left fold}
and the interpretation (\ref{fold-right}) is called \emph{right fold}.

Of course, if the operator $\circ$ is associative, the interpretation
is (by definition) insignificant from the denotational point of
view (although the amount of resources used by those interpretations
might differ under various circumstances).

Also, it is easy to see that, for the above $fold$ operation
to make sense, it must be the case that $\circ : A \times A \mapsto A$.

This requirement can be loosened a bit, though. By introducing
additional argument $e$, we allow the operator to have a type
$B \times A \mapsto B$ in the case of the left fold, or
$A \times B \mapsto B$ in the case of the right fold: then,
\emph{fold-left}$(\circ, e, [a_1, a_2, ..., a_n])$ is interpreted
as $((...((e \circ a_1) \circ a_2)\circ ...)\circ a_n)$,
and \emph{fold-right}$(\circ, e, [a_1, a_2, ..., a_n])$
is interpreted as $(a_1 \circ (a_2 \circ (... \circ (a_n \circ e) ...)))$.

The purpose for the presence the additional argument $e$ can be explained
as follows. Imagine that you have a state of the world, $S$, and a
list $L$ (possibly infinite) of actions that occur in the order as they
appear on the list. There's an update function called $\lhd$ that takes
a state and an action and returns an updated state. Under such circumstances,
the evolution of the world can be modeled as \emph{fold-left}$(\lhd, S, L)$.

If the operation $\circ$ has a neutral element (i.e. an element
$\mathbbm{1}$ such that, for any valid $x$, $\mathbbm{1} \circ x = x$ in the
case of left fold, or $x \circ \mathbbm{1} = x$ in the case of right
fold), it is often convenient to choose it as the argument $e$.

\subsubsection{Pattern matching}

We can write programs that operate on s-expressions using the
primitive functions \texttt{pair?}, \texttt{car}, \texttt{cdr}
and \texttt{eq?}. For example, if we want to know whether a given
s-expression \texttt{exp} represents a sum of exactly two elements,
say, \texttt{(+~a~b)} (for some \texttt{a} and \texttt{b}),
we can write a compound condition

\begin{Snippet}
(and (pair? exp)
     (eq? (car exp) '+)
     (pair? (cdr exp))
     (pair? (cdr (cdr exp)))
     (eq? (cdr (cdr (cdr exp))) '()))
\end{Snippet}

If we would like to capture the first and second operand to plus,
using, say, the \texttt{let} from, we would obtain:

\begin{Snippet}
(if (and (pair? exp)
         (eq? (car exp) '+)
         (pair? (cdr exp))
         (pair? (cdr (cdr exp)))
         (eq? (cdr (cdr (cdr exp))) '()))
    (let ((a (car (cdr exp)))
          (b (car (cdr (cdr exp)))))
       ;; here a and b are bound to the first
       ;; and the second operand of +
       ...)
    ;;else
    ...)
\end{Snippet}

However, it is very difficult to analyze this sort of code
(whose task is rather simple). It is therefore convenient to
extend the syntax of Scheme with the \texttt{match} macro\footnote{
  This document uses a subset of the syntax proposed in
  \cite{WrightCartwright1997}. A portable implementation
  of this pattern language can be obtained from
  \url{http://synthcode.com/scheme/match.scm}.
}, so that the expression
\begin{Snippet}
(match exp
  (('+ a b)
   ;; here a and b are bound to the first
   ;; and the second operand of +
    ...)
  ;; possibly other matches go next
  ...)
\end{Snippet}

expands to the above condition and binding list. Furthermore,
we can establish a convention, that whenever a compound expression
appears in the place of an argument in the \texttt{lambda}
form (or any derivative form, such as \texttt{let} or \texttt{let*}),
it gets pattern-matched, so for instance
\texttt{(lambda\,\,((a\,\,.\,\,b))\,\,body)}
would be interpreted as
\texttt{(lambda\,\,(x)\,\,(match\,\,x\,\,((a\,\,.\,\,b) body)))}
(where the variable \texttt{x} does not occur free in the \texttt{body}
form). The \texttt{\_} (underscore) symbol has a special meaning:
it matches anything, but does not get bound to any value.

We will sometimes be making use of a bit more exotic (and less obvious)
feature of the pattern matcher, namely -- the \texttt{...} (ellipsis)
operator. It behaves similarly to the \texttt{,@} (\texttt{unquote-splicing})
operator in that it ``unsplices'' a list:

\begin{Snippet}
  (match '(1 2 3 4 5)
    ((x y ... z)
     ;; x is bound to 1,
     ;; y -- to the list (2 3 4),
     ;; and z -- to 5
    ...))
\end{Snippet}

\section{The meta-circular evaluator}

The subset of Scheme that we chose for the purpose of this work
is powerful enough to express itself \footnote{As it was noted
  by Christian Queinnec, ``literature about Lisp rarely resists
  that narcissistic pleasure of describing Lisp in Lisp''\cite{Queinnec1996},
  and here we make no exception.}

Since in this setup Scheme is used as both the language that we
talk about and the language that we use to describe that language,
this construct is called a \textit{meta-circular evaluator}, as
Scheme becomes a \textit{metalanguage} for \textit{itself}.

As a result, everything that has been said in this chapter about
the semantics of Scheme, can be worded more precisely and rigorously
using a formal notation.

In order to make the code clearer, we shall restrict the object
language even more. First, we shall assume, that the \texttt{define}
form does not appear in our program. Indeed, a mechanical procedure
for transforming a program containing the \texttt{define} forms
into a program that only uses \texttt{lambda}, \texttt{let} and
\texttt{let*} (and a fixed-point combinator) is given in the appendix
\ref{Y-lining}.

Also, for the sake of brevity, we shall ignore the issues
concerning user-defined syntactic extensions. We shall also
assume that the programs do not reuse either of the syntactical
keywords (like \texttt{quote} or \texttt{lambda} or \texttt{if}),
so for example we shall not be concerned with programs such as
\begin{Snippet}
  ((lambda (lambda)
      (lambda lambda))
   (lambda (quote)
      (quote quote)))
\end{Snippet}

There is a general technique commonly referred to as
\textit{$\alpha$-conversion}\cite{SussmanSteele1976} for dealing with
cases like this (where each of the variables bound with the
\texttt{lambda} form or its derivative is renamed before the evaluation).

Lastly, we will assume that no derivative forms such as \texttt{let},
\texttt{let*}, \texttt{match}, \texttt{quasiquote}, \texttt{and}
or \texttt{or} are used in the evaluated program, i.e. that all such
forms have been expanded prior to the evaluation. The appendix \ref{expander}
explains in detail how these derivative forms can be converted
to primitive forms.

%%%%%%%%%%%%%%%%%%%%%%%%%%%%%%%%%%%%%%%%%%%%%%%%%%%%%%%%%%%%%%%%%%%%%%%

\subsection{The core evaluator}

The core of our evaluator is the \texttt{value} function, which
produces the value of a given expression. It consists of a case
analysis over the structure of expression. In particular,
the ability to create new procedures using the \texttt{lambda}
form plays the key role for the language's expressive power.

In order to manipulate the procedures, we need to come
up with some way of representing them. Since the only
types of objects that are available in our language are
symbols, numbers, pairs and procedures, we'd need
to form our representation from some combination
of them\footnote{There would be very little point in using procedures
to represent procedures, as the main purpose of our
evaluator is to explain how the procedures work -- therefore,
we ought to express compound procedure application without
resorting to primitive procedure application (unless we apply
primitive procedures)}.

We decided to represent procedures as tagged list, whose first
element is a special symbol \texttt{procedure-tag}\footnote{
Ideally, user should not be able to input the tag in one's own code,
as it could disrupt the operation of the evaluator, but
in practice that doesn't matter much, because the semantics of
Scheme does not allow to apply lists, so we can choose an
arbitrary symbol as the tag, and the only undesired consequence
is that the evaluator will try to evaluate expressions of the form
\texttt{(('procedure-tag other-data ...)\,\,arguments)
}}:

\begin{Snippet}
(define (value expression environment)
  (match expression
    (('lambda args body)
     `(procedure-tag ,args ,body ,environment))
\end{Snippet}
\begin{Snippet}    
    (('quote literal)
     literal)
\end{Snippet}
\begin{Snippet}        
    (('if condition consequent alternative)
     (if (value condition environment)
         (value consequent environment)
     ;else
         (value alternative environment)))
\end{Snippet}
\begin{Snippet}    
    ((operator . operands)
     (let ((procedure (value operator environment))
           (arguments (map (lambda (operand)
                              (value operand environment))
			   operands)))
         (application procedure arguments)))
\end{Snippet}
\begin{Snippet}    
    (_
     (cond ((symbol? expression)
            (lookup expression environment))
           ((number? expression)
            expression)
           (else
            (error 'unrecognized-expression expression))))
    ))
\end{Snippet}

The most important thing that is left is to explain how
the procedure application occurs: first, we should extend
the original environment of the procedure with the values
of arguments, and then execute the program defined in
the body of the \texttt{lambda} form. (The only
exception concerns the primitive functions, which are
going to be handled using the primitive \texttt{apply})

\begin{Snippet}
(define (application procedure arguments)
  (match procedure
    (('procedure-tag parameters body closure)
     (let ((environment `(,(tie parameters arguments) . ,closure)))
       (value body environment)))
    (_ ; a primitive procedure
     (apply procedure arguments))))
\end{Snippet}

\subsection{Representing environments}

As it was suggested in the \texttt{application} procedure, we represent
the environment as a list of \emph{frames}, where each frame is a list
of two-element \texttt{(key\,\,value)} lists\footnote{
  Traditionally, frames would rather be represented as lists of dotted
  pairs of the form \texttt{(key\,\,.\,\,value)} (so-called \textit{assoc lists}),
  because it would allow to represent a single entry using a single cons
  cell, whereas the \texttt{(key value)} list normally uses two cons
  cells. However, since data structure optimization is the core theme
  of this work, we don't value such arguments too much. We believe that
  improper lists should only be used for constructing and destructuring
  lists, and that actual data should only be stored using proper lists
  (unless the data stored represents the Scheme source. However, even
  on such occasions we could treat the dot as a special symbol, rather
  than a part of syntax).
  Note also, that another popular representation of frames consists
  of two lists, where the first contains symbols' names, and the
  second consists of the corresponding values, for example
  \texttt{((a b c) (1 2 3))}. This representation is often called
  \textit{a rib cage}\cite{Dybvig1987}.
}.

\begin{Snippet}
(define (lookup-frame name frame)
  (match frame
    (((key value) . rest)
     (if (equal? key name)
       `(,key ,value)
     ;else
       (lookup-frame name rest)))
    (()
     #false)))
\end{Snippet}
\begin{Snippet}    
(define (lookup name environment)
  (let (((frame . frames) environment))
    (match (lookup-frame name frame)
      ((key value)
       value)
      (_
       (lookup name frames)))))
\end{Snippet}

The \texttt{tie} function ties the parameter names with their
values, extending a given environment. Although it is not required
by the Scheme standard, the \texttt{tie} function can be defined
to support destructured bindings:

\begin{Snippet}
(define (tie names values)
  (match names
    (()
     '())
    ((name . other-names)
     (let (((value . other-values) values))
       `(,@(tie name value) . ,(tie other-names other-values))))
    (rest
     `((,rest ,values)))))
\end{Snippet}

The initial environment should contain all the primitive functions
available in the language. In the simplest case, it should be sufficient
to define it as

\begin{Snippet}
  (define initial-environment
    `(((cons ,cons)
       (car ,car)
       (cdr ,cdr)
       (eq? ,eq?)
       (pair? ,pair?)
       (number? ,number?)
      ;; ...
      )))
\end{Snippet}

to use the underlying representation of primitive data structures.
In the following chapter we will try to explain how the
primitive data structures can be implemented.
